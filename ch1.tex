
\chapter{BASICS OF GRAVITATIONAL WAVES}

The direct detection of gravitational waves (GWs) promises to usher in a new era of astronomy. The GW spectrum represents an entirely new window on the universe, independent of, and complimentary to, electromagnetic (EM) radiation. Gravitational waves can be used to directly probe objects unobservable by EM telescopes; e.g., the properties of black holes, the equation of state of neutron stars, and the state of the universe prior to the emission of the cosmic microwave background. Joint GW and EM observations offer more possibilities, such as understanding the progenitors of short-hard gamma-ray bursts (GRBs) and measuring the expansion of the universe. The GW spectrum would also give us insight into the physics of strong field gravity and numerical solutions of the Einstein equations, as well as provide a test for alternative theories of gravity . The U.S. Laser Interferometer Gravitational-wave Observatory (LIGO) and the French-Italian Virgo interferometer are seeking to make the first direct detections of gravitational waves . To date, LIGO has completed six Science runs. The first five of these runs were known as initial LIGO. In LIGO’s fifth science run (S5), which lasted from November 2005 to September 2007, the LIGO detectors reached their design sensitivity, as they were sensitive to gravitational waves with strain amplitudes of ∼ 10−21 in the 40−7000 Hz frequency band .LIGO’s sixth science run (S6), also known as enhanced LIGO , lasted from July 2009 until October 2010. Hardware improvements were make to the detectors for S6; during this period the LIGO detectors met and exceeded the sensitivity of S5. Virgo has had three science runs. Virgo’s first science run (VSR1) overlapped with S5, lasting from May 2007 until October


\section{What are  Gravitationl Waves?}
Gravitational waves are "ripples" in space time caused by some of the most violent and energetic processes in the Universe. Albert Einstein predicted the existence in his general theory of relativity .
His mathematics show that the massive accelerating objects (Black holes or the nuetron stars  ) would spacetime in such a way that 'waves' of undulating space time propagate in all the direction away from the source.These Ripples travel in speed of light in space, carry the information about their origin and clues to the nature of gravity itself.Gravitational waves represent the transport of gravitational energy through space and are solutions 
of the linearized Einstein equations. They can be explained using mechanical analogues and are the result of the 
interaction between gravity and light. Gravitational waves have created a new window in astronomy world and are 
considered as important tools for testing theories of astrophysics and cosmology. They can also be used to test the 
nature of general relativity and have revolutionized the understanding of compact objects and gravity itself.
The first direct detection of gravitational waves occurred in 2015 when LIGO observed the merger of two black holes. 
This groundbreaking discovery provided strong evidence for the existence of gravitational waves and opened a new 
era in astrophysics. Gravitational waves travel at the speed of light and have a unique polarization that causes 
oscillations in spacetime in two orthogonal directions. The waves themselves are stretches and compressions of 
spacetime. Gravitational waves are extraordinarily weak. When they pass through the Earth, the stretching and 
squeezing of spacetime is incredibly tiny, on the order of a fraction of the diameter of an atomic nucleus. Detecting 
such small changes requires extremely precise instruments. The Laser Interferometer Gravitational-Wave Observatory 
(LIGO) and Virgo are observatories specifically designed to detect the  Gravitational Waves.



% TODO: \usepackage{graphicx} required
\begin{figure}
	\centering
	\includegraphics[width=0.7\linewidth]{"C:/Users/uakhi/OneDrive/Desktop/image/Artists-Impression-of-Colliding-Black-Holes 1"}
	\caption{Black Hole merge LIGO }
	\label{fig:artists-impression-of-colliding-black-holes-1}
\end{figure}



\section{The Newtionian gravity to Einstein}

Newton's theory of gravity has enjoyed great success in describing many aspects of our every-day life and additionally explains most of the motions of celestial bodies in the Universe. He developed the set of equation that described the physical properties of the universe in exact manner. These equations were very successful in the classical world. In the time of 19th century ,people start noticing that not all plays according to this rule and this was the time of extensive study of the phenomena of electricity,magnetism and light. Maxwell published the st of equations that combined all these phenomena into a singke peice called "Electromagnetism". soon after Maxwell's discovery ,people realised that there is something wrong when its come to the equations.Their form changes when we move from one inertial frame to another . so an individual who is not moving can observe distinctively different physical phenomena that a person who is moving. All the beauty observed from the Newtonian physics was gone wrong. Then ,a new mathematical transformation was discovered "Lorentz Transformation". 
The "Lorentz Transformation" was different from the standard transformation of inertial frames that had been used in the newtonian physics.which say that the length and time do actually change , depending on which frame of reference you are in.
 This got Einstein wondering whether the transformation that preserved the structure of Maxwell’s equations was merely a mathematical trick or whether there was something fundamental about it. He wondered whether time and space were absolute, or whether the principle of invariance of the laws of physics should be paramount.
In 1905, Einstein decided that it is the invariance of the laws of physics that should have the highest status, and postulated the principle of relativity: that all inertial frames are equivalent, the observer’s motion (with constant velocity) is irrelevant, and that all laws of physics should have the same form in all inertial frames. When combined with electromagnetism, this principle would require that the transformation from one inertial frame to another must have a structure of the Lorentz transformation, meaning that time and space are no longer absolute and change their properties when changing from one inertial frame to another. In 1907 Einstein realised that his theory was not complete. The principle of relativity was only applicable to observers moving with a constant velocity. It also did not fit with the Newtonian description of gravity.
The road to general relativity
All this reasoning convinced Einstein that the geometry of the spacetime and the physical processes that take place in the spacetime, are related to each other and that one can affect the other. It also led to a striking conclusion: what we perceive as gravity is just a consequence of the motion through the spacetime. The larger the curvature of the spacetime the stronger gravity is. It took Einstein eight years to find the relation between the geometry of spacetime and physics.

The equations that he presented in 1915 not only led to a completely different interpretation of events around us but also provided an explanation for some baffling or yet to be discovered phenomena: from the anomalous orbit of the planet Mercury, through the bending of light by the Sun’s gravity, to predicting the existence of black holes and expanding universe.
It was a bumpy road from Newtownian physics to special and then general relativity. But each step, driven by Einstein’s insight, drove inexorably towards a picture of the universe that persists to this day.

\section{Einstine Gravity}

 Einstein imagined moving alongside a 
light wave at the speed of light. He realized that in this frame of reference the light wave 
would appear to freeze. The electric and magnetic fields would oscillate in time , but he 
wouldn't see the wave moving away from him. The problems associated with this "frozen 
wave" picture led Einstein to the key insight that the speed of light is the same for all 
observers, independent of their state of motion. He also assumed that the laws of physics 
must be the same for all observers.
Einstein elevated these two assumptions to physical principles, called the principles of
special relativity. They apply to situations in which observers or objects are moving at a 
constant speed and direction. His resulting theory, the Special Theory of Relativity , though 
inconsistent with our common-sense notions of the world, has been shown to be entirely 
consistent to extremely high accuracy, with all experimental tests. When Albert Einstein 
published his first paper on special relativity in 1905, he did not yet fully grasp its 
geometrical nature. He came to appreciate these aspects of the theory over the next several 
years while working with mathematician Hermann Minkowski (1864–1909). Minkowski 
developed the idea of space time, in which events occur in a four-dimensional “space” that 
includes both(the three dimensions of ) space and (one more dimension of ) time—hence, 
the name. 
In 1915, Einstein published a new work on a General Theory of Relativity, a new 
mathematical formalism to extend his Special Theory to situations in which objects were 
able to accelerate. General relativity gives the equation for calculating the curvature of the space time given by

\begin{equation}
	G_{mv}=\frac{8piG}{c^4}T_{mv}
\end{equation}


Where G is the universal gravitational constant.The equation is called Einstein field 
equation in general relativity.When there is no mass,energy momentum is zero so the 
curvature is zero and space is flat. The General Theory has wide applicability to massive objects in the 
Universe. In particular, it can be applied to objects that appeared to violate the predictions 
made using Newton’s simpler formalism, such as the orbital movement of the planet 
mercury.



\subsection*{What are the Predictions?}

To understand Einstein’s prediction of gravitational waves, let’s revisit Newton’s Law of 
Gravitation . There is nothing in this law that describes how the effects of gravity are 
transmitted from one place to another. In fact, according to this law, if we move one of the 
masses to a different point in space, then the other mass “knows” this instantly, and it reacts 
accordingly. This ability of gravity to act instantly across any distance puzzled many 
contemporaries of Newton. It also violates Einstein’s Special Theory of Relativity, as it is 
not possible for any signal or information to cross space faster than light can travel.In
general relativity, gravity is not a force as envisioned by Newton. Instead, gravity is the
result of spacetime being distorted, and the distortions of spacetime are caused by
the distribution of mass and energy. Changing the mass-energy distribution will generally
change the spacetime curvature, and thus also the gravitational effects of objects in
that spacetime. However, when a mass distribution is changed, the effects of that change
are not carried instantaneously to all points in space. The information is carried at the speed
of light by ripples in the spacetime fabric – small stretches and compressions in the
coordinates describing the spacetime. These ripples are called gravitational waves. 


\section{How are Gravitational Waves Generated}
Technically speaking, every physical object that accelerates produces gravitational waves. This includes humans, cars, airplanes etc. But the masses and accelerations of objects on Earth are far too small to make gravitational waves big enough to detect with our instruments. To find big enough gravitational waves, we have to look far outside of our own solar system.
The Universe is filled with incredibly massive objects undergoing rapid accelerations that generate gravitational waves that we can now detect. Known objects are  pairs of black holes or neutron stars orbiting each other, or a neutron star and black hole orbiting each other or gigantic stars blowing themselves up at the ends of their lives. Astronomers have defined four categories of gravitational waves based on what object or system generates the waves: Continuous, Compact Binary Inspiral, Stochastic, and Burst. Each category of objects generates a characteristic set of gravitational wave signals .



% TODO: \usepackage{graphicx} required
\begin{figure}
	\centering
	\includegraphics[width=0.7\linewidth]{C:/Users/uakhi/OneDrive/Desktop/image/MassPlot_graveyard_190521}
	\caption{GW190521}
	\label{fig:massplotgraveyard190521}
\end{figure}


\subsection*{Type of the Gravitational Wave }

\subsection{Compact Binary Inspiral Gravitational Waves}
The first class of gravitational waves LIGO is hunting for is Compact Binary Inspiral gravitational waves. So far, all of the objects LIGO has detected fall into this category. Compact binary inspiral gravitational waves are produced by orbiting pairs of massive and dense ("compact") objects like black holes and neutron stars. There are three subclasses of "compact binary" systems in this category:
Binary Neutron Star (BNS) - two neutron stars orbiting each other.
Binary Black Hole (BBH) - two black holes orbiting each other.
Neutron Star-Black Hole Binary (NSBH) - a neutron star and a black hole orbiting each other.



% TODO: \usepackage{graphicx} required
\begin{figure}
	\centering
	\includegraphics[width=0.7\linewidth]{C:/Users/uakhi/OneDrive/Desktop/image/GW170104SpiralOrbSchem}
	\caption{compact}
	\label{fig:gw170104spiralorbschem}
\end{figure}


Each binary pair creates a unique pattern of gravitational waves that depends on, among other things, the masses of each object, how their orbits are oriented with respect to the Earth, and how far away they are, but the mechanism of wave-generation is the same across all three. It is called "inspiral".

Inspiral occurs over millions of years as pairs of dense compact objects revolve around each other. As they orbit, they radiate gravitational waves that carry away some of the system's orbital energy. Over  millennia, this causes the objects to move closer and closer together. The closer they are, the faster they orbit each other, which causes them to radiate stronger gravitational waves, which causes them to lose more orbital energy, inch ever closer, orbit faster, lose more energy, move closer, orbit faster... etc. The objects are doomed, inescapably locked in a runaway accelerating spiraling embrace.Each binary pair creates a unique pattern of gravitational waves that depends on, among other things, the masses of each object, how their orbits are oriented with respect to the Earth, and how far away they are, but the mechanism of wave-generation is the same across all three. It is called "inspiral".

Inspiral occurs over millions of years as pairs of dense compact objects revolve around each other. As they orbit, they radiate gravitational waves that carry away some of the system's orbital energy. Over  millennia, this causes the objects to move closer and closer together. The closer they are, the faster they orbit each other, which causes them to radiate stronger gravitational waves, which causes them to lose more orbital energy, inch ever closer, orbit faster, lose more energy, move closer, orbit faster... etc. The objects are doomed, inescapably locked in a runaway accelerating spiraling embrace.
\subsection{Continuous Gravitational Waves}
% TODO: \usepackage{graphicx} required
\begin{figure}
	\centering
	\includegraphics[width=0.7\linewidth]{C:/Users/uakhi/OneDrive/Desktop/image/artist_NSIllustration_CREDIT__NSF_LIGO_Sonoma_State_University_A._Simonnet}
	\caption{continuous Gravitational waves}
	\label{fig:artistnsillustrationcreditnsfligosonomastateuniversitya}
\end{figure}

Continuous gravitational waves are expected to be produced by a single spinning massive object like a neutron star. Any bumps on or imperfections in the spherical shape of this star will generate gravitational waves as it spins. If the spin-rate of the star stays constant, so too will the gravitational waves it emits. That is, the gravitational wave is continuously the same frequency and amplitude (like a singer holding a single note). That's why these are called “Continuous Gravitational Waves”.


\subsection{Stochastic Gravitational Waves}

Astronomers predict that there are so few significant sources of continuous or binary inspiral gravitational waves in the Universe that LIGO doesn't worry about the possibility of more than one passing by Earth at the same time (producing confusing signals in the detectors). However, we do presume that many small gravitational waves are passing by from all over the Universe all the time, and that they are mixed together at random. These small waves from every direction make up what is called a “Stochastic Signal”, so called because the word 'stochastic' means having a random pattern that may be analyzed statistically but not predicted precisely. These will be the smallest and most difficult gravitational waves to detect, but it is possible that at least part of this stochastic signal may originate from the Big Bang. Detecting relic gravitational waves from the Big Bang will allow us to see farther back into the history of the Universe than ever before.
\subsection{Burst Gravitational Waves.}
Burst gravitational waves come from short-duration unknown or unanticipated sources. Searching for these kinds of signals requires being utterly open-minded. LIGO's data analysis tools must recognize a pattern of gravitational wave signals even if such a pattern has not been modeled (i.e., predicted from theory) before. If you don’t know what you’re looking for, it’s really hard to find it! While this makes searching for burst gravitational waves difficult, detecting them has the greatest potential to reveal revolutionary information about the Universe.

% TODO: \usepackage{graphicx} required
\begin{figure}
	\centering
	\includegraphics[width=0.7\linewidth]{C:/Users/uakhi/OneDrive/Desktop/image/neutronstar_CASEY_REED_PENN_STATE}
	\caption{}
	\label{fig:neutronstarcaseyreedpennstate}
\end{figure}


The search for 'burst gravitational waves' is truly a search for the unexpected—both because LIGO has yet to detect them, and because there are still so many unknowns that we really don’t know what to expect! For example, we may not know enough about the physics of a system to predict how gravitational waves from such a source will appear. 

Of course, we may also detect gravitational waves from systems we never knew about before. To search for these kinds of gravitational waves, we cannot assume that they will have well-defined properties like the ones LIGO scientists have previously modeled. This means we cannot restrict our analyses to searching only for the signatures of gravitational waves that scientists have predicted.
\section{How are they detected}

The detection of gravitational waves has been a long-standing challenge for researchers. The first direct evidence of gravitational waves was observed in 2015 by the LIGO and Virgo collaborations, confirming the existence of these waves and the merger of a binary black hole system. LIGO (Laser Interferometer Gravitational-Wave Observatory) and Virgo use laser interferometry to measure tiny changes in the lengths of their arms caused by passing gravitaional waves. Each observatory consists of multiple kilometers - long arms arranged in an L-shape. When a gravitational wave passes through the observatories, it causes a slight stretching and squeezing of spacetime.
% TODO: \usepackage{graphicx} required
\begin{figure}
	\centering
	\includegraphics[width=0.7\linewidth]{C:/Users/uakhi/OneDrive/Desktop/image/ligo-livingston-sec-exhibit-2}
	\caption{}
	\label{fig:ligo-livingston-sec-exhibit-2}
\end{figure}
This alters the lengths of the arms very slightly. The laser interferometer detects these minute changes in distance. The returning laser beams create an 
interference pattern when they recombine. This pattern is analyzed to determine whether the lengths of the arms have changed. Gravitational waves manifest as oscillations in this interference pattern. In 2017, LIGO and Virgo detected a gravitational wave signal, GW170817, associated with the merger of two neutron stars. This event was particularly significant because it was also observed across the electromagnetic spectrum, marking the first-ever multi-messenger observation of a cosmic event. The successful detection of gravitational waves, as achieved by LIGO and Virgo, represents a significant milestone in astrophysics, opening up a new era of observational astronomy.
% TODO: \usepackage{graphicx} required
\begin{figure}
	\centering
	\includegraphics[width=0.7\linewidth]{C:/Users/uakhi/OneDrive/Desktop/image/ligo20160211a}
	\caption{}
	\label{fig:ligo20160211a}
\end{figure}




























